\documentclass[12pt, letterpaper, titlepage]{article}
\usepackage[left=3.5cm, right=2.5cm, top=2.5cm, bottom=2.5cm]{geometry}
\usepackage[MeX]{polski}
\usepackage[utf8]{inputenc}
\usepackage{graphicx}
\usepackage{enumerate}
\usepackage{amsmath} %pakiet matematyczny
\usepackage{amssymb} %pakiet dodatkowych symboli
\title{Dokument o roślinach w Wiedzminie 3}
\author{Marcin Michoń}
\date{15.10.2022}
\begin{document}
\maketitle
\section{Marcin} %%powiedź coś o osobie
Nazywam się Marcin Michoń, mieszkam w Nidzicy i zajmuje się wieloma rzeczami:
\begin{enumerate}[i)]
\item studiuje na kierunku informatycznym
\item lubię grać w piłkę
\item grać na gitarze
\item słuchać muzyki
\item Oraz grać w gry \textbf {\textit {komputerowe}}
\end{enumerate}
\newpage
W ostatnim czasie zanurzając się w rozgrywce w grze Wiedźmin 3, zaskoczyła mnie różnorodność wielu gatunków kwiatów. Na tyle się nimi zachwyciłem aby przytoczyć kilka z nich.

\section{Ogólnie rośliny} %%Powiedź coś o roślinach o gólnie
W Wiedźminie trzecim występuje wiele gatunków kwiatów i roślin.
W grze rośliny pełnią wiele funkcji.  Oczywiście zjedzenie borówek spowoduje regeneracje punktów życia jednak to nie wszystko. Flora może służyć do tworzenia trunków, rynsztunku, służą do wykonania zadań ale również w pewnych miejscach pełnią funkcje wrogów z  którymi powinniśmy się uporać, jednak w tym dokumencie skupie się na wytwarzaniu jak i krótkim opisie.
\newline 


\subsection{Arenaria}
Roślina zielna o krwistoczerwonych jagodach, rośnie w dzikich środowiskach\\

\subsubsection{wytwarzanie}
\begin{enumerate}
\item Biała Mewa
\item Biały barwnik do zbroi
\item Eliksir Filtr Petriego
\item Jad Wisielców
\item Odwar z biesa
\item Odwar z katakana
\item Olej przeciw upiorom
\item Ulepszony Filtr Petriego
\item Ulepszony Jad Wisielców
\item Ulepszony olej przeciw drakonidom
\item Ulepszony olej przeciw trupojadom
\item Ulepszony olej przeciw upiorom
\item Wyśmienity Filtr Petriego
\item Wyśmienity olej przeciw ogrowatym
\item Wyśmienity olej przeciw trupojadom
\end{enumerate} 
\newpage
\subsection{Dmuchawiec}
Roślina wspominana przez Pani Jeziora, rośnie w dzikich środowiskach
\subsubsection{wytwarzanie}
\begin{enumerate}
\item Czerwony mutagen
\item Odwar z archegryfa
\item Odwar z bazyliszka
\item Odwar z mglaka
\item Odwar z upiora
\item Olej przeciw trupojadom
\item Ulepszona Bomba dwimerytowa
\item Ulepszona Wilga
\item Ulepszony Kartacz
\item Ulepszony Księżycowy Pył
\item Ulepszony olej przeciw trupojadom
\item większy czerwony mutagen
\item Wyśmienita Wilga
\item Żółty barwnik do zbroi
\end{enumerate}
\newpage
\subsection{ToJad}
Ziele, łagodzi objawy likantropii, a zasuszony w pęczkach odstrasza wilkołaki. Występuje także w grach Wiedźmin, Wiedźmin 2: Zabójcy Królów i Wiedźmin 3: Dziki Gon.\\
Był jednym ze składników eliksiru, którego Geralt użył podczas walki ze Strzygą. Po wysuszeniu służył do odstraszania wilkołaków. Wykorzystywany także do podtrzymywania ognia. Tojad jest silnie działającym lekarstwem, ale w dużych dawkach był jedną z najsilniejszych trucizn. Można kupić jak i zebrać, rośnie w dzikich środowiskach
\subsubsection{wytwarzanie}
\begin{enumerate}
\item Pełnia
\item Fioletowy barwnik do zbroi
\item Mikstura na przeczyszczenie
\item Olej przeciwko przeklętym
\item Ulepszona Pełnia
\item Ulepszony olej przeciw wampirom
\item Ulepszony olej przeciw przeklętym
\item Ulepszony Puszczyk
\item Wyśmienity olej przeciw przeklętym
\item Wyśmienity olej przeciw wampirom
\item Wyśmienita Pełnia
\item Wyśmienity Puszczyk

\end{enumerate}
\newpage
\section{zakończenie}
Rośliny w grach pełnią wiele funkcji, służą między innymi do wytwarzania ale prawdziwym ,życiu również możemy wykorzystywać kwiaty np. do tworzenia \textbf{leków} czy \textbf{jedzenia}.
\end{document}